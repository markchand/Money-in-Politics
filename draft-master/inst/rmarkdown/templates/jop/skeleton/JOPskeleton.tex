% Options for packages loaded elsewhere
\PassOptionsToPackage{unicode}{hyperref}
\PassOptionsToPackage{hyphens}{url}

\documentclass{jopsubmission}

%% Math support
\usepackage{amssymb} % Note: loads amsfonts
\usepackage{amsmath}

\usepackage[T1]{fontenc}
\usepackage[utf8]{inputenc}
\usepackage{textcomp} % provide euro and other symbols

% Use upquote if available, for straight quotes in verbatim environments
\IfFileExists{upquote.sty}{\usepackage{upquote}}{}
\IfFileExists{microtype.sty}{% use microtype if available
  \usepackage[]{microtype}
  \UseMicrotypeSet[protrusion]{basicmath} % disable protrusion for tt fonts
}{}

\usepackage{float} % Noorah says this will help with \newpage in References 
\usepackage{xcolor}

\IfFileExists{xurl.sty}{\usepackage{xurl}}{} % add URL line breaks if available

\IfFileExists{bookmark.sty}{\usepackage{bookmark}}{\usepackage{hyperref}}
\hypersetup{
          hidelinks,
  pdfcreator={LaTeX via pandoc}}

\urlstyle{same} % disable monospaced font for URLs


%% For tables
\usepackage{booktabs}

%% For figures
\usepackage{graphicx}
\usepackage{adjustbox,lipsum} % Alternate to \resizebox, maybe? 
\usepackage{grffile}

\setlength{\emergencystretch}{3em} % prevent overfull lines
\providecommand{\tightlist}{%
  \setlength{\itemsep}{0pt}\setlength{\parskip}{0pt}}



% We don't want a date in there
\date{\vspace{-5ex}}

    \title{Campaign Contribution Limits and Corruption\\
Evidence from the 50 states}


        
% We need to remove section numbering
\setcounter{secnumdepth}{-\maxdimen}

\begin{document}

    \maketitle
            \begin{abstract}
        Since its 1976 \emph{Buckley v. Valeo} decision, the Supreme Court has
        allowed only two potential justifications for limiting campaign
        donations: To prevent corruption or the appearance of corruption. Do
        campaign finance limits actually mitigate corruption or its appearance?
        As current challenges to contribution limits wind their way through
        federal courts, the answer to this question will inform the Supreme
        Court's decision to uphold or strike down remaining limits on political
        campaign contributions. Using state-level data from 1990-2012 on
        contribution limits, corruption convictions, and media mentions of
        corruption, this paper finds associations between corruption convictions
        and the presence of limits on donations to political action committees
        (PACs), the amount of those limits, and the presence of limits on
        donations to parties. Individual donation limits, however, do not prove
        associated with either measure of corruption.
        \end{abstract}
    

\newpage 

\subsection{Introduction}\label{introduction}

In 1976, the Supreme Court ruled in \emph{Buckley v. Valeo} that
political campaign expenditures are the equivalent of free speech, and
therefore protected by the first amendment. At the same time, the Court
laid out one mechanism and two permissible reasons for limiting campaign
contributions that enable those expenditures. First, it argued that more
money contributed does not amount to ``more'' speech, allowing for the
use of campaign contribution dollar amount limits. Donating one dollar
or \$5,000 to a political campaign amounts to a speech act, the Court
held, and so it is permissible to impose limits on the amount of
contributions as long as contributions are not prevented. Second, having
laid out this mechanism by which money in politics might be limited, the
court laid out only two constitutionally permissible reasons for
limiting contributions: To prevent corruption or the appearance thereof.
As Kang (2012) argues, the holdings in \emph{Buckley v. Valeo} provided
``the basic framework of contribution limits and disclosure
requirements'' for the following four decades. Since \emph{Buckley v.
Valeo}, Congress has passed only one campaign finance law, the
Bipartisan Campaign Reform Act of 2002. Better know as McCain-Feingold,
the bill took aim at unlimited ``soft money'' contributions to political
parties and expanded the Federal Election Commission's disclosure
requirements to what it called ``electioneering communications,''
political ads not paid for by campaigns themselves.

Apart from McCain-Feingold, for forty years all changes in the landscape
of campaign finance have come from the Supreme Court and appellate
courts, which have worked to balance questions of corruption with those
of free speech. In 1981, the Court struck down contribution limits
related to ballot measures (\emph{Citizens against Rent Control v. City
of Berkeley}). Just before the passage of McCain-Feingold, the courts
considered state-level limits, allowing low limits in Missouri
(\emph{Nixon v. Shrink Missouri}, 2000), though it would later finding
Vermont's to be low enough that they had a meaningful impact on
political speech (\emph{Randall v. Sorrell}, 2006).

The passage of the McCain-Feingold in 2002 inspired a wave of challenges
to campaign contributions and expenditures, driven mostly by
conservative and religious political groups. In 2007, the Court examined
the McCain-Feingold ban on airing issue advertisements close to
election. Banning such ads, the Court argued, would have a chilling
effect on free speech, and it struck down the ban ( \emph{FEC v.
Wisconsin Right to Life}). Then, in 2010, the Court issued what has
become one of its most well-known rulings: In \emph{Citizens United v.
FEC} (2010), it struck down limits on independent expenditures by
outside groups, including corporations and unions. One critical aspect
of \emph{Citizens United} is the establishment of the principle that,
according to Justice Anthony Kennedy, ``independent expenditures do not
lead to, or create the appearance of, quid pro quo corruption.'' The
Court's emphasis on quid pro quo corruption as the only justification
for contribution limits extended the following year, as it struck down
an Arizona effort to level the political playing field with matching
state campaign funds (\emph{Arizona Free Enterprise Club v. Bennett},
2011).

Since 2011, the network of conservative legal groups behind Citizens
United has continued to take aim at contribution limits as an
unnecessary restriction of political speech, with a focus on individual
contribution limits. The 1971 Federal Election Campaign Act placed a
limit on the amount an individual could give to one campaign, in
addition to two-year limits on the total amount an individual could give
to political campaigns. In \emph{McCutcheon v. FEC} (2014) James Bopp,
one of the attorneys behind the Citizens United challenge, convinced the
Court of the unconstitutionality of total individual limits. Another
case, \emph{Holmes v. FEC} (2017), targeted the distribution of
individual campaign contribution across primary and general elections.
In yet another, \emph{Lair v. Mangan} (2018), Bopp once again took aim
at individual contribution limits, arguing that the state of Montana had
not shown compelling evidence that campaign contribution limits
decreased the risk of corruption.

The Ninth Circuit Court of Appeals upheld Montana's campaign
contribution limits, and the Fifth Circuit similarly upheld limits in
Austin, TX (\emph{Zimmerman v. City of Austin}, 2018). But further
challenges to contribution limits are likely. Bopp said that he planned
to appeal the Ninth Circuit's ruling. Even if that fails, Bopp and other
conservative groups plan to carry on. ``I handle a lot of cases,'' he
said after the 2014 McCutcheon ruling, ``and I`m not done yet.'' Supreme
Court Justice Clarence Thomas laid the groundwork for future challenges
in his commentary on the McCutcheon decision by arguing that more
dollars did, in fact, equate to more speech. Once established, that
principle supports the argument that campaign contribution dollar amount
limits are a potentially unconstitutional restriction on speech (Hurley
\& Debenedetti 2014, Zall 2017).

\subsection{Prior Work}\label{prior-work}

Since the Buckley v. Valeo decision, a wave of scholarship has taken
advantage of the campaign contribution data collected by the Federal
Election Commission to examine the causes and consequences of political
campaign contributions, including whether contributions buy access (not
to a meaningful degree, find Welch 1982 and Langbein 1986) and the
effectiveness of contributions relative to lobbying efforts (yes,
according to Wright 1990). Since the 2010 Citizens United and
speechnow.org decisions, another arm of scholarship has examined the
flows of money to groups engaging in independent expenditures (Spencer
and Wood 2014, Werner and Coleman 2014).

But lobbying, money for access, and independent expenditures all fall
outside of the narrow definition of quid pro corruption that the Supreme
Court considers a compelling enough state interest to justify campaign
contribution limits. The scholarly research on corruption thus narrowly
defined is mixed and heated. Stratmann finds evidence that donations
determined votes in at least one instance (1991); he also finds that the
timing of political action committees (PAC) donations suggests that
corporations believe those donations do affect legislators' votes
(1998). Reacting to the absence of evidence that donations lead to
favorable final votes on bills, Hall and Wayman (1990) look for and find
a positive relationship between political donations and the amount of
attention that legislators give to a particular issue. Other scholars,
however, are not so convinced. Wright finds that once lobbying efforts
are factored in, the effects of contributions on voting outcomes washes
out (1990). And it is notoriously difficult to tease out whether
political donations change legislators' minds or instead flow to
legislators with established, friendly positions (Ansolabehere et al
2003).

As mentioned above, the Court is also interested in the appearance of
corruption. On this score, results are more consonant: There seems to be
little association between campaign spending and public trust in
government (Primo 2002). Other factors, including media coverage of
campaign finance law and the professionalization of legislatures, are
more predictive of public perceptions of corruptions (Rosenson 2009).
And while the public does seem to report viewing campaign finance as a
contributor to corruption, there is not evidence that reform efforts
will have an effect on those views (Persily \& Lammie 2004).

One fundamental limitation of these federal level studies on campaign
contributions and corruption is the absence of a counter-factual. Given
that contribution limits have been in place since the 1970s, it is
difficult to argue what might have happened in their absence. Recent
work on campaign finance has attempted to address this problem by
focusing on the effects of differences in campaign finance laws at the
state level. Variation across states and over time allows room for
studying the causes and consequences of campaign finance limits,
separate from the question of corruption. Studies suggest that campaign
finance limits are associated with less electoral competition (Gross et
al 2002, Lott 2006), but that disclosure rules and some organizational
limits are associated with greater levels of government trust (Primo and
Milyo 2006) and may increase the return on campaign expenditures
(Stratmann 2006). As mentioned, Michael Barber's 2016 work finds that
lower limits on PAC donations relative to individual donations are
associated with higher polarization.

A largely separate body of work has examined the causes and consequences
of corruption at the state level, separate from campaign finance. Most
use the Department of Justice's Annual Public Integrity Report, which
correlate highly with state level perceptions of corruption (Walker and
Calcagno 2011, Glaeser and Saks 2006, Liu \& Mikesell 2014). Using that
data, researchers have found evidence that political corruption leads to
inflated state budgets and that the presence of casinos and subsequent
corruption (Liu \& Mikesell 2014, Walker \& Calcagno 2013). They have
also shown the absence of a link between state ethics commissions and
reduced corruption (Crider and Milyo 2013).

In response to criticisms of corruption convictions as a proper measure
of corruption, researchers have also developed alternative measures of
state level corruption. Boylan and Long make use of surveys of state
reporters (2003). Cordis and Milyo deploy the recently available
Transactional Records Access Clearinghouse (TRAC) database of political
corruption (2013). Dincer and Jonhston introduce a Corruption
Reflections Index, which tracks mentions of corruption in Associated
Press reports (2016).

Fewer scholars have begun connecting these bodies of work to examine the
impact of state-level campaign finance regulations on measures of quid
pro quo corruption--one of only two questions of import to the Supreme
Court as it considers the constitutionality of contribution limits. The
exception to this is a 2013 working paper by Cordis and Milyo, which
campaign contribution limits have little effect on public perceptions of
corruption. This paper seeks to address this gap in research, building
on Cordis and Milyo's work putting state level campaign finance data in
direct conversation with state level corruption data.

\subsection{Data and Design}\label{data-and-design}

In this paper, I build upon the growing body of state-level political
science research to ask whether campaign contribution limits at the
state level associated with lower levels of corruption. Given open
debate about the proper measure of state-level corruption, I use two
measures. The first, a Corruptions Convictions Index, is the traditional
measure of state level corruption, built from Department of Justice
convictions data. The second, the Corruptions Reflection Index, is a new
measure put forward by Dincer and Johnston (2016). I combine these
measures with a third data set from Michael Barber's 2016 work on
campaign contribution limits and political polarization; and a fourth
data set from James Alt and David Dryer Lassen, to provide controls for
the full model (Alt and Lassen 2012).

Barber's data set is a panel of contribution limits from 1990-2012 for
the lower house of every state legislature. Barber compiled this data
set from the FEC, Westlaw, and the Nation Council of State Legislatures.
Each dollar amount is inflation indexed to 2010 dollar amounts, and
bi-annualized. There are three different types of limits in the Barber
database: Individual contribution limits, individual-to-party limits,
and PAC limits. Barber utilized individual and PAC contributions in his
analysis of polarization, leaving out individual-to-party limits given
that party contributions typically make up less than 10\% of total
political fundraising; I include it this analysis to determine whether
parties, rather than PACs or individuals, might be the locus of
corruption as measured by our other two data sets (Barber 2016).

% \begin{adjustbox}{max width=\textwidth}
\begin{table}[!htbp]\centering 
\caption{Descriptive Statistics for State Level Campaign Finance Limits} 
\label{} 
\resizebox{\textwidth}{!}{%
\begin{tabular}{@{\extracolsep{5pt}}lccccccc} 
\\[-1.8ex]\hline 
\hline \\[-1.8ex] 
Statistic & \multicolumn{1}{c}{N} & \multicolumn{1}{c}{Mean} & \multicolumn{1}{c}{St. Dev.} & \multicolumn{1}{c}{Min} & \multicolumn{1}{c}{Pctl(25)} & \multicolumn{1}{c}{Pctl(75)} & \multicolumn{1}{c}{Max} \\ 
\hline \\[-1.8ex] 
Individual Limits & 802 & 3,801 & 4,236 & 143 & 827 & 5,078 & 23,164 \\ 
Individual-to-Party Limits & 472 & 33,222 & 45,732 & 834 & 7,243 & 32,882 & 196,470 \\ 
Party Limits & 543 & 23,804 & 32,616 & 246 & 5,246 & 24,976 & 150,904 \\ 
PAC Limits & 742 & 6,396 & 9,440 & 223 & 1,231 & 9,302 & 96,940 \\ 
\hline \\[-1.8ex] 
\multicolumn{8}{l}{Notes: 'N' equals number of state-years (out of n=1150) in which that type of limit is present} \\ 
\end{tabular} 
}
\end{table}
% \end{adjustbox}
    

The second data set comes from the Department of Justice's ``Report to
Congress on the Activities and Operations of the Public Integrity
Section'' by way of the Institute for Corruption Studies.\footnote{N.B.
  According to one of the researchers'
  \href{https://about.illinoisstate.edu/odincer/Documents/CVDincerApril2016.docx}{curricula
  vitae}, the Institute for Corruption Studies is funded in part by a
  grant from the Charles Koch Research Foundation. Organizations funded
  by Charles and David Koch are also funders of legal groups driving
  legal battles mentioned in the introduction, according to a
  \href{https://www.publicintegrity.org/2017/11/15/21279/kochs-key-among-small-group-quietly-funding-legal-assault-campaign-finance}{report}
  by the Center for Public Integrity.} Since 1976, the FBI has tracked
political corruption at the state level, and the Corruption Convictions
Index (CCI) used here includes the number of corruption convictions in a
year normalized by the population in each state. This measure of
corruption is widely used in academic research, though not without
criticism. Descriptive statistics for corruption convictions are
included below.

The third data set is the Institute for Corruption Studies' Corruption
Reflections Index (CRI), put together by Dincer and Johnston for their
2016 article on state-level corruption, and which they create in
response to the limitations of the Corruption Convictions. The
Corruption Reflections Index represents the fraction of Associated Press
stories about politics in a particular state that mention political
corruption (Dincer and Johnston 2016).

\begin{table}[!htbp] \centering 
  \caption{Descriptive Statistics for Corruption and Reflections} 
  \label{} 
  \resizebox{\textwidth}{!}{%
\begin{tabular}{@{\extracolsep{5pt}}lccccccc} 
\\[-1.8ex]\hline 
\hline \\[-1.8ex] 
Statistic & \multicolumn{1}{c}{N} & \multicolumn{1}{c}{Mean} & \multicolumn{1}{c}{St. Dev.} & \multicolumn{1}{c}{Min} & \multicolumn{1}{c}{Pctl(25)} & \multicolumn{1}{c}{Pctl(75)} & \multicolumn{1}{c}{Max} \\ 
\hline \\[-1.8ex] 
Corruption Convictions Index (CCI) & 1,999 & 2.96 & 2.99 & 0.00 & 1.03 & 3.97 & 33.85 \\ 
Corruption Reflections Index (CRI) & 1,887 & 0.28 & 0.19 & 0.00 & 0.15 & 0.37 & 2.16 \\ 
\hline \\[-1.8ex] 
\end{tabular} 
}
\end{table}

The fourth data set provides controls for the panel model. In their 2012
work examining variation in prosecutorial resources, Alt and Lassen
follow Glaeser and Saks (2006) in controlling for income and education
levels, size of government, legislature control, and urbanization (see
Appendix A). Alt and Lassen's data set runs from 1976-2004. When
included in the second and third models, this limits the analysis to the
years from 1990-2004 and n=720.

This analysis has its limitations, beginning with the data sets
themselves. Barber's state campaign contributions data lacks information
about upper houses of state legislatures, which are included in both the
CCI and CRI. The Corruption Convictions Index suffers from a number of
issues, as outlined by Dincer and Johnston in their introduction to the
Corruption Reflections Index. They argue that the CCI includes only
federal data; that the resources available to prosecutors varies from
state to state; that US Attorneys are political appointees; and that the
measure records only convictions, which is at best a reflection of
underlying corruption.

The Corruption Reflections Index, too, has its flaws: It relies on a
single news source, the Associated Press. Similar to federal
prosecutors' offices, the resources available to each AP office vary. It
also demonstrates high variation in the data; Utah, for example, has a
mean CRI score of 0.46 and standard deviation of 0.38, in a data set
with an overall mean of 0.28 and standard deviation of 0.19. The
Corruption Reflections Index also seems to assume in its name that media
reports of corruption are a mere reflection of corruption, rather than
an endogenous variable in the ecosystem that enables corrupt behavior.
Previous work, in contrast, has argued that media have a clear bias and
play an important role in shaping public perceptions of corruption,
rather than just reflecting it (Sorauf 1987; Ansolabehere, Snowberg and
Snyder 2005).

Most broadly, this analysis does not directly address the ``appearance
of corruption,'' the second of the Supreme Court's reasons for allowing
campaign contribution limits; to do so effectively would require time
series, state-level data specifically about quid pro quo corruption. It
is also the case that this analysis looks to the states intentionally in
order to address the issue of the counter-factual inherent in looking at
federal data, it may not be the case that statewide results are
generalizable to, or distinct from, what is occurring at the federal
level. Third, it is also the case that campaign finance laws may not be
equally enforced across states, even though the enforcement measures
used here are from federal rather than state authorities. Finally, the
period in question (1990-2004) is short enough that there may be some
factor that influences corruption arrest across a meaningful portion of
the data set.

\subsection{Methods}\label{methods}

Having constructed a full 1990-2012 panel data set from these three data
sets, I then run a series of time series regressions testing the
relationship between contribution limits and both dependent variables,
the Corruption Convictions Index and the Corruption Reflections Index.

I include three study variables in both sets of models: Individual
contribution limits, individual-to-party contribution limits, and PAC
contribution limits. For each of these independent variables, I
construct two variables for use in the models. The first is a dummy
variable indicating whether there are contribution limits in that state
in that year, coded as ``0'' if there are no limits and ``1'' if there
are. The second is a continuous variable indicating the dollar amount of
the limit in the state in that year. By interacting these two variables,
I test two separate associations. The dummy variable tests whether there
is a difference in the CCI and CRI in states with limits versus states
without limits. The continuous variable tests whether higher or lower
limits are associated with higher or lower CCI and CRI scores, within
the group of states that have limits in a given year. After performing
this transformation on all three independent variables, I test two
initial sets of models: (1) Corruption Convictions Index versus campaign
contribution limits in a given state-year, and (2) Corruption
Reflections Index versus campaign contributions in a given state-year.

I build a model of political corruption using control variables
previously demonstrated by Alt and Lassen to be correlated with
corruption convictions, resulting in a data set covering 1990-2004, the
period of overlap among the data sets listed above. These include
controlling for the size and education levels of the population, the
ideology of a population, income levels and inequality, education
levels, the number and relative income of government employees, and
urbanization. Building a model of corruption that includes variables
previously demonstrated to be correlated with corruption levels, then
adding in campaign finance data, I can better identify the association
of campaign finance laws with corruption levels. For each of these
models, I test fixed state and year effects and random effects. I also
test appropriate transformations of both dependent and independent
variables. The models are as follows:

\paragraph{Corruption Convictions Index versus campaign contribution
limits in a given
state-year:}\label{corruption-convictions-index-versus-campaign-contribution-limits-in-a-given-state-year}

\((log)Convictions = \alpha_{state} + \\ \hspace{3.2cm} \beta_{1}{indiv.dummy} + \beta_{2}{indiv.dummy}\times{indiv.limit} + \\ \hspace{3.2cm} \beta_{3}{pac.dummy} + \beta_{4}{pac.dummy}\times{pac.limit} + \\ \hspace{3.2cm} \beta_{5}{indiv-to-party.dummy} + \beta_{6}{indiv-to-party.dummy}\times{indiv-to-party.limit} + \\ \hspace{3.2cm} Controls\times\gamma + \epsilon\)

\paragraph{Corruption Reflections Index versus campaign contributions in
a given
state-year:}\label{corruption-reflections-index-versus-campaign-contributions-in-a-given-state-year}

\((log)Reflections = \alpha_{state} + \\ \hspace{3.1cm} \beta_{1}{indiv.dummy} + \beta_{2}{indiv.dummy}\times{indiv.limit} + \\ \hspace{3.1cm} \beta_{3}{pac.dummy} + \beta_{4}{pac.dummy}\times{pac.limit} + \\ \hspace{3.1cm} \beta_{5}{indiv.to.party.dummy} + \beta_{6}{indiv.to.party.dummy}\times{indiv.to.party.limit} + \\ \hspace{3.1cm} Controls\times\gamma + \epsilon\)

\subsection{Results}\label{results}

The results of the panel regressions reveal multiple correlations
between state-level campaign limits and Corruption Convictions; and one
correlation between those limits and the Corruption Reflections Index.

The first set of models (Table 3) measures the association of various
campaign contribution limits with the Corruption Convictions Index. The
first model includes only control variables; the second model adds in
campaign finance limit variables. The third layers in fixed effects at
the state level, fixed time effects, and logged study variables; the
fourth removes fixed effects but maintains the logged variables. Not
included here are: models with random effects, models with unlogged
dependent variables, and models with other combinations of fixed effects
and variable transformations. The resulting models identify correlations
between the CCI and the existence individual-to-party limits: The
presence of such limits is associated with \emph{greater} corruption as
measured by convictions. In Model 2, with unlogged independent
variables, both the presence and amount of limits on PAC contributions
to campaigns is associated with lower levels of corruption measured in
terms of convictions.

The second set of models (Table 4) walks through the same progression of
models, measuring the association of various campaign contribution
limits and the Corruption Reflections Index (CRI). Model 4, which
includes logged independent and dependent variables but not fixed
effects, demonstrates that the presence of individual-to-party limits is
associated with lower levels of corruption as measured by media mention
of corruption. As with the CCI models in Table 3, the presence and
amount of PAC limits, when unlogged, are associated with lower levels of
corruption reflections.

\begin{table}[!htbp] \centering 
  \caption{Models of Contribution Limits and Corruption Convictions} 
  \label{} 
\small 
\resizebox{\textwidth}{!}{%
\begin{tabular}{@{\extracolsep{-15pt}}lcccc} 
\\[-1.8ex]\hline 
\hline \\[-1.8ex] 
 & \multicolumn{4}{c}{\textit{Dependent variable:}} \\ 
\cline{2-5} 
\\[-1.8ex] & \multicolumn{4}{c}{Corruption Convictions Index (logged)} \\ 
\\[-1.8ex] & (1) & (2) & (3) & (4)\\ 
\hline \\[-1.8ex] 
 Individual limit (amount) &  & 0.00 &  &  \\ 
  &  & (0.00) &  &  \\ 
  Individual limit (logged) &  &  & 0.14 & 0.05 \\ 
  &  &  & (0.09) & (0.04) \\ 
  Individual limit (dummy) &  & 0.69 & 0.47 & 0.12 \\ 
  &  & (0.79) & (0.38) & (0.16) \\ 
  Individual-to-party limit (amount) &  & 0.00 &  &  \\ 
  &  & (0.00) &  &  \\ 
  Individual-to-party limit (logged) &  &  & 0.02 & 0.04 \\ 
  &  &  & (0.07) & (0.03) \\ 
  Individual-to-party limit (dummy) &  & 0.19$^{**}$ & 0.19 & 0.16$^{*}$ \\ 
  &  & (0.09) & (0.21) & (0.09) \\ 
  PAC limit (amount) &  & $-$0.00$^{**}$ &  &  \\ 
  &  & (0.00) &  &  \\ 
  PAC limit (logged) &  &  & $-$0.09 & $-$0.06 \\ 
  &  &  & (0.06) & (0.04) \\ 
  PAC limit (dummy) &  & $-$0.99$^{**}$ & $-$0.13 & $-$0.14 \\ 
  &  & (0.41) & (0.24) & (0.13) \\ 
  Divided government & $-$0.14$^{***}$ & $-$0.13$^{***}$ & $-$0.09$^{*}$ & $-$0.14$^{***}$ \\ 
  & (0.05) & (0.05) & (0.05) & (0.05) \\ 
  AUSAs per population (MMs) & 0.03$^{***}$ & 0.03$^{***}$ & 0.03$^{**}$ & 0.03$^{***}$ \\ 
  & (0.00) & (0.01) & (0.01) & (0.01) \\ 
  Percent high school grads & $-$2.63$^{***}$ & $-$2.62$^{***}$ & 0.47 & $-$2.73$^{***}$ \\ 
  & (0.61) & (0.62) & (1.67) & (0.63) \\ 
  Gov revenues per capita (1000s) & 0.00$^{***}$ & 0.00$^{**}$ & 0.00$^{*}$ & 0.00$^{**}$ \\ 
  & (0.00) & (0.00) & (0.00) & (0.00) \\ 
  Urbanization & $-$0.00$^{*}$ & $-$0.00$^{**}$ & $-$0.02 & $-$0.00$^{**}$ \\ 
  & (0.00) & (0.00) & (0.03) & (0.00) \\ 
  Log of population (MMs) & 0.22$^{***}$ & 0.21$^{***}$ & 2.08$^{***}$ & 0.22$^{***}$ \\ 
  & (0.03) & (0.04) & (0.72) & (0.04) \\ 
  Constant & 1.67$^{***}$ & 2.01$^{**}$ &  & 1.44$^{**}$ \\ 
  & (0.30) & (0.87) &  & (0.59) \\ 
 \hline \\[-1.8ex] 
  & Base Model & Full Model & Fixed Effects & Pooled \\ 
Observations & 671 & 671 & 671 & 671 \\ 
R$^{2}$ & 0.13 & 0.14 & 0.03 & 0.14 \\ 
Adjusted R$^{2}$ & 0.12 & 0.13 & $-$0.08 & 0.12 \\ 
F Statistic & 16.60$^{***}$ (df = 6; 664) & 9.23$^{***}$ (df = 12; 658) & 1.76$^{*}$ (df = 12; 598) & 8.92$^{***}$ (df = 12; 658) \\ 
\hline 
\hline \\[-1.8ex] 
\textit{Note:}  & \multicolumn{4}{r}{$^{*}$p$<$0.1; $^{**}$p$<$0.05; $^{***}$p$<$0.01} \\ 
\end{tabular} 
}
\end{table}

\begin{table}[!htbp] \centering 
  \caption{Models of Contribution Limits and Corruption Reflections} 
  \label{} 
\small 
\resizebox{\textwidth}{!}{%
\begin{tabular}{@{\extracolsep{-15pt}}lcccc} 
\\[-1.8ex]\hline 
\hline \\[-1.8ex] 
 & \multicolumn{4}{c}{\textit{Dependent variable:}} \\ 
\cline{2-5} 
\\[-1.8ex] & \multicolumn{4}{c}{Corruption Reflections Index (logged)} \\ 
\\[-1.8ex] & (1) & (2) & (3) & (4)\\ 
\hline \\[-1.8ex] 
 Individual limit (amount) &  & 0.00 &  &  \\ 
  &  & (0.00) &  &  \\ 
  Individual limit (logged) &  &  & $-$0.00 & 0.01 \\ 
  &  &  & (0.02) & (0.01) \\ 
  Individual limit (dummy) &  & 0.20 & $-$0.01 & 0.03 \\ 
  &  & (0.17) & (0.07) & (0.03) \\ 
  Individual-to-party limit (amount) &  & 0.00 &  &  \\ 
  &  & (0.00) &  &  \\ 
  Individual-to-party limit (logged) &  &  & 0.01 & 0.00 \\ 
  &  &  & (0.01) & (0.01) \\ 
  Individual-to-party limit (dummy) &  & $-$0.03$^{*}$ & $-$0.05 & $-$0.05$^{***}$ \\ 
  &  & (0.02) & (0.04) & (0.02) \\ 
  PAC limit (amount) &  & $-$0.00$^{**}$ &  &  \\ 
  &  & (0.00) &  &  \\ 
  PAC limit (logged) &  &  & $-$0.01 & $-$0.01 \\ 
  &  &  & (0.01) & (0.01) \\ 
  PAC limit (dummy) &  & $-$0.18$^{**}$ & $-$0.03 & $-$0.01 \\ 
  &  & (0.09) & (0.04) & (0.03) \\ 
  AUSAs per million population & 0.00$^{**}$ & 0.00$^{***}$ & $-$0.00 & 0.00$^{***}$ \\ 
  & (0.00) & (0.00) & (0.00) & (0.00) \\ 
  Citizen ideology measure & $-$0.00$^{***}$ & $-$0.00$^{***}$ & $-$0.00 & $-$0.00$^{***}$ \\ 
  & (0.00) & (0.00) & (0.00) & (0.00) \\ 
  Percent high school graduates & $-$0.42$^{***}$ & $-$0.35$^{***}$ & $-$0.54$^{*}$ & $-$0.38$^{***}$ \\ 
  & (0.13) & (0.13) & (0.31) & (0.13) \\ 
  Real per capita gov revenues (1000s) & 0.00$^{***}$ & 0.00$^{***}$ & 0.00$^{**}$ & 0.00$^{***}$ \\ 
  & (0.00) & (0.00) & (0.00) & (0.00) \\ 
  Inequality: Male wages & $-$0.01$^{***}$ & $-$0.00$^{**}$ & $-$0.00 & $-$0.00$^{**}$ \\ 
  & (0.00) & (0.00) & (0.00) & (0.00) \\ 
  Urbanization & 0.00$^{***}$ & 0.00$^{***}$ & $-$0.01 & 0.00$^{***}$ \\ 
  & (0.00) & (0.00) & (0.00) & (0.00) \\ 
  Log of population (millions) & 0.03$^{***}$ & 0.02$^{***}$ & $-$0.04 & 0.03$^{***}$ \\ 
  & (0.01) & (0.01) & (0.13) & (0.01) \\ 
  Constant & 0.30$^{***}$ & 0.22 &  & 0.22$^{*}$ \\ 
  & (0.07) & (0.20) &  & (0.13) \\ 
 \hline \\[-1.8ex] 
  & Base Model & Full Model & Fixed Effects & Pooled \\ 
Observations & 672 & 672 & 672 & 672 \\ 
R$^{2}$ & 0.21 & 0.24 & 0.04 & 0.24 \\ 
Adjusted R$^{2}$ & 0.20 & 0.23 & $-$0.08 & 0.22 \\ 
F Statistic & 24.92$^{***}$ (df = 7; 664) & 16.19$^{***}$ (df = 13; 658) & 1.71$^{*}$ (df = 13; 598) & 15.61$^{***}$ (df = 13; 658) \\ 
\hline 
\hline \\[-1.8ex] 
\textit{Note:}  & \multicolumn{4}{r}{$^{*}$p$<$0.1; $^{**}$p$<$0.05; $^{***}$p$<$0.01} \\ 
\end{tabular} 
}
\end{table}

\subsection{Discussion}\label{discussion}

On their face, the models above support two tentative conclusions and
present one puzzle: First, in none of the specifications above are
limits on individuals' campaign contributions correlated with either
corruption convictions or corruption reflections. This evidence of
absence is not absence of evidence, of course, but consistent with the
hypothesis that individual donations are not a primary source of
political corruption.

The second tentative conclusion is that there is a yet-unspecified
relationship between limits on PAC contributions and corruption. This is
true in models in which \emph{unlogged} PAC limits are included in the
models, but not when that variable, with a skew of 3.468, is logged.
Given the non-normality of that variable, even when logged, it may be
that a different transformation would more accurately specify these
relationships. If those relationships hold, two possibilites may explain
why limiting PAC donations might be associated with lower levels of
corruption convictions and media coverage when individual limits are
not. First, it may be that actors engaged enough in the political
process to be able to corrupt public officials are likely to be entwined
with, and work through, political action committees. Second, it may be
that when corruption is routed through an institution such as a PAC, the
number of people involved may make it more likely to be discovered,
prosecuted and reported.

The puzzle in this data is the varying effects of limits on
individual-to-party donations. The amount of those limits has no
relationship in these models to either corruption convictions or
reflections. But the presence of a limit is associated with
\emph{higher} levels of corruption convictions and \emph{lower} levels
of media coverage of corruption.\footnote{These relationships remain
  robust to changes in model specification, including when other
  donation limit variables are removed from the model.} This suggests
that states with limits on individual donations to political parties see
more corruption, but less reporting on those convictions.

What are the legal and policy implications of these results? The Supreme
Court presently equates political donations with political speech,
meaning that it is protected by the first amendment, but that preventing
corruption or its appearance in the political system is so important
that first amendment rights can be abridged to ensure it. For the Court,
no other justification--including creating a more informed electorate,
or stimulating more competitive elections, or equality of financial
participation in elections--is an important enough concern that we
should limit political speech in the form of campaign donations. The
results above suggest that state-level restrictions on donations from
individuals may not be demonstrably associated with lower corruption,
but that other types of donation limits (on contributions from PACs to
campaigns, and from individuals to parties) may indeed be associated
with lowering corruption. Such results might allow for courts to be more
open to the possibility of restricting donations to and from organized
groups such as PACs and parties, without having to restrict individual
donations and participation in the electoral process.

\newpage

\subsection{Appendix A: Descriptions and Descriptive Statistics of
Control
Variables}\label{appendix-a-descriptions-and-descriptive-statistics-of-control-variables}

In their 2012 paper ``Enforcement and Public Corruption: Evidence from
the American States,'' Glaeser and Saks make use of the following
control variables, which we have included here:

\begin{itemize}
\tightlist
\item
  \emph{AUSAs per million population}, where the number of general
  attorneys in the U.S. Attorney's office serves as a proxy for the
  number of federal resources focused on corruption in a particular
  state
\item
  \emph{Relative government wages}, the wages of government workers
  relative to that of the state in which they live, under the hypothesis
  that lower relative wages will make government workers more prone to
  corruption
\item
  \emph{Divided government}, where legislature and executive are
  controlled by different parties, \emph{Binding one-term limits} and
  \emph{Binding two-term limits}, which Alt and Lassen call ``shadow of
  the future'' variables
\item
  \emph{Real per capita income (\$1000)}, \emph{Unemployment},
  \emph{Citizen ideology measure}, \emph{Percent high school graduates},
  \emph{Real per capita gov revenues (\$1000)}, \emph{Inequality: Male
  wages}, \emph{Urbanization} and \emph{Log of population (millions)},
  which are included as social and demographic controls
\end{itemize}


\begin{table}[H]\centering % Changed !htbp to H per Noorah's suggestion 
  \caption{Descriptive Statistics of Control Variables} 
  \label{} 
\small 
\resizebox{\textwidth}{!}{%
\begin{tabular}{@{\extracolsep{-5pt}}lccccccc} 
\\[-1.8ex]\hline 
\hline \\[-1.8ex] 
Statistic & \multicolumn{1}{c}{N} & \multicolumn{1}{c}{Mean} & \multicolumn{1}{c}{St. Dev.} & \multicolumn{1}{c}{Min} & \multicolumn{1}{c}{Pctl(25)} & \multicolumn{1}{c}{Pctl(75)} & \multicolumn{1}{c}{Max} \\ 
\hline \\[-1.8ex] 
Divided government & 1,392 & 0.45 & 0.50 & 0 & 0 & 1 & 1 \\ 
Binding one-term limit & 1,392 & 0.06 & 0.23 & 0 & 0 & 0 & 1 \\ 
Binding two-term limit & 1,392 & 0.21 & 0.41 & 0 & 0 & 0 & 1 \\ 
Real per capita income (1000s) & 1,392 & 13,117.77 & 2,830.47 & 7,707.26 & 11,042.72 & 14,894.11 & 23,280.97 \\ 
AUSAs per million population & 1,344 & 12.36 & 5.61 & 2.18 & 8.21 & 15.34 & 36.53 \\ 
Unemployment & 1,296 & 5.88 & 2.01 & 2.30 & 4.50 & 6.90 & 17.40 \\ 
Citizen ideology measure & 1,392 & 46.94 & 15.05 & 8.45 & 36.12 & 56.86 & 95.97 \\ 
Percent high school graduates & 1,392 & 0.48 & 0.08 & 0.27 & 0.42 & 0.54 & 0.62 \\ 
Real per capita gov revenues (1000s) & 1,392 & 1,666.10 & 459.65 & 825.12 & 1,327.91 & 1,957.08 & 4,184.86 \\ 
Inequality: Male wages & 1,392 & 17.08 & 3.72 & 8.24 & 14.45 & 19.38 & 31.42 \\ 
Relative government wages & 1,392 & 0.10 & 0.01 & 0.08 & 0.09 & 0.10 & 0.12 \\ 
Urbanization & 1,392 & 67.14 & 21.25 & 24.99 & 48.80 & 83.66 & 100.00 \\ 
Log of population (millions) & 1,392 & 1.19 & 0.99 & $-$0.93 & 0.47 & 1.79 & 3.57 \\ 
\hline \\[-1.8ex] 
\end{tabular} 
}
\end{table}

\newpage

\subsection{References}\label{references}

Ansolabehere, Stephen, John M de Figueiredo, and James M Snyder Jr.
2003. ``Why is there so Little Money in US Politics?'' Journal of
Economic Perspectives. 17(1): 105--130.

Alt, J. E., \& Lassen, D. D. (2012). Enforcement and public corruption:
evidence from the American states. The Journal of Law, Economics, and
Organization, 30(2), 306-338.

Barber, Michael J. ``Ideological Donors, Contribution Limits, and the
Polarization of American Legislatures.'' The Journal of Politics 78.1
(2016): 296-310.

Boylan, R. T., \& Long, C. X. (2003). Measuring public corruption in the
American states: A survey of state house reporters. State Politics \&
Policy Quarterly, 3(4), 420-438.

Cordis, A., \& Milyo, J. (2013). Do state campaign finance reforms
reduce public corruption?. George Mason University, Mercatus Center,
Working Paper, (13-09).

Cordis, A. S., \& Milyo, J. (2016). Measuring public corruption in the
United States: evidence from administrative records of federal
prosecutions. Public Integrity, 18(2), 127-148.

Crider, K., \& Milyo, J. (2013). Do State Ethics Commissions Reduce
Political Corruption: An Exploratory Investigation. UC Irvine L. Rev.,
3, 717.

Dincer, O., \& Johnston, M. (2016). Political Culture and Corruption
Issues in State Politics: A New Measure of Corruption Issues and a Test
of Relationships to Political Culture. Publius: The Journal of
Federalism, 47(1), 131-148.

Glaeser, E. L., \& Saks, R. E. (2006). Corruption in america. Journal of
public Economics, 90(6), 1053-1072.

Gross, D. A., Goidel, R. K., \& Shields, T. G. (2002). State campaign
finance regulations and electoral competition. American Politics
Research, 30(2), 143-165.

Hall, R. L., \& Wayman, F. W. (1990). Buying time: Moneyed interests and
the mobilization of bias in congressional committees. American political
science review, 84(3), 797-820.

Hurley, L. \& Debenedetti, G. (2014, April 04). Legal victory for
big-money campaign donors to be felt in states, courts. Retrieved
December 16, 2017, from
\url{https://www.reuters.com/article/us-usa-court-election-analysis/legal-victory-for-big-money-campaign-donors-to-be-felt-in-states-courts-idUSBREA3308620140404}

Langbein, L. I. (1986). Money and access: Some empirical evidence. The
journal of politics, 48(4), 1052-1062.

Liu, C., \& Mikesell, J. L. (2014). The impact of public officials'
corruption on the size and allocation of US state spending. Public
Administration Review, 74(3), 346-359.

Lott, J. R. (2006). Campaign finance reform and electoral competition.
Public Choice, 129(3-4), 263-300.

Persily, N., \& Lammie, K. (2004). Perceptions of corruption and
campaign finance: When public opinion determines constitutional law.
University of Pennsylvania Law Review, 119-180.

Primo, D. M. (2002). Public opinion and campaign finance: Reformers
versus reality. The independent review, 7(2), 207-219.

Primo, D. M., \& Milyo, J. (2006). Campaign finance laws and political
efficacy: evidence from the states. Election Law Journal, 5(1), 23-39.

Rosenson, B. A. (2009). The effect of political reform measures on
perceptions of corruption. Election Law Journal, 8(1), 31-46.

Spencer, Douglas M., and Abby K. Wood. 2014. ``Citizens United, states
divided: an empirical analysis of independent political spending.'' LJ
89: 315.

Stratmann, Thomas. 1991. ``What Do Campaign Contributions Buy?
Deciphering Causal Effects of Money and Votes.'' Southern Economic
Journal. Pp.606-620.

Stratmann, T. (1998). The market for congressional votes: Is timing of
contributions everything?. The Journal of Law and Economics, 41(1),
85-114.

Stratmann, T. (2006). Contribution limits and the effectiveness of
campaign spending. Public Choice, 129(3), 461-474.

Walker, D. M., \& Calcagno, P. T. (2013). Casinos and political
corruption in the United States: A Granger causality analysis. Applied
Economics, 45(34), 4781-4795.

Walker, D. M., \& Jackson, J. D. (2011). The effect of legalized
gambling on state government revenue. Contemporary Economic Policy,
29(1), 101-114.

Werner, Timothy, and John J. Coleman. 2014. ``Citizens United,
independent expenditures, and agency costs: Reexamining the political
economy of state antitakeover statutes.'' The Journal of Law, Economics,
\& Organization. 127-159.

Welch, W. P. (1982). Campaign contributions and legislative voting: Milk
money and dairy price supports. Western Political Quarterly, 35(4),
478-495.

Zall, B. (2017, October 24). Ninth Circuit Tees Up Latest Challenge to
Citizens United and McCutcheon. Retrieved November 22, 2017, from
\url{https://publicpolicylegal.com/2017/10/24/ninth-circuit-tees-up-latest-challenge-to-citizens-united-and-mccutcheon/}


\end{document}

